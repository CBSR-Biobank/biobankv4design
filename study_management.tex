\chapter{Study Management}

\section{Entities}

\begin{description}[listparindent=\parindent]

  \item[\entitytarget{Study}] \hfill \\ Represents a collection of participants and
    specimens collected for a particular research study. The study is part of
    an aggregate that define the types of specimens to be collected and how
    they are to be collected from participants.

    A study can be enabled or disabled. When disabled, changes to its
    configuration are possible but patients and specimens cannot be added. When
    enabled, no further configuration changes are allowed. However,
    participants and specimens can be added.

    A study may have one or more comments assigned to it during it's lifetime.

  \item[\entitytarget{CollectionEventType}] \hfill \\ A classification name,
    unique to the \entitylink{Study}, of a visit by the study's participants.

  \item[\entitytarget{ProcessingType}] \hfill \\ Describes a regularly
    performed procedure with a unique name (within its
    \entitylink{Study}). There should be one or more associated
    \entitylink{SpecimenProcesingLinkType}s that (1) further define legal
    procedures and (2) allow logging of procedures performed on different types
    of \entitylink{Specimen}s.

  \item[\entitytarget{}] \hfill \\

  \item[\entitytarget{}] \hfill \\


\end{description}

\section{Value Objects}

\begin{description}

  \item[\valobjtarget{SpecimenGroupCollectionEventType}] \hfill \\ Defines
    which types of specimens (i.e. which \valobjlink{SpecimenGroup}s) need to
    be collected as part of a \entitylink{CollectionEventType}.

  \item[\valobjtarget{SpecimenLinkType}] \hfill \\ Represents a regularly
    performed processing procedure involving two \entitylink{Specimen}s: an
    input, which must be in a specific \valobjlink{SpecimenGroup}, and an
    output, which must be in a specific \valobjlink{SpecimenGroup}.

  \item[\valobjtarget{SpecimenGroup}] \hfill \\ Ownership, summary, storage,
    and classification information that applies to an entire group or
    collection of \entitylink{Specimen}s.

  \item[\valobjtarget{SpecimenType}] \hfill \\ A standardised set of
    classifications that describe \emph{what} a \entitylink{Specimen}
    is. Potential examples include: urine, whole blood, plasma, nail, protein,
    etc.

  \item[\valobjtarget{Preservation}] \hfill \\ Describes how a
    \entitylink{Specimen} should be preserved/stored by describing temperature
    requirements, as well as a preservation method (see
    \valobjlink{PreservationType}).

  \item[\valobjtarget{PreservationType}] \hfill \\ A standardised set of
    methods for preserving and storing \entitylink{Specimen}s.  Potential
    examples include: frozen specimen, RNA later, fresh specimen, slide, etc.

  \item[\valobjtarget{}] \hfill \\

  \item[\valobjtarget{}] \hfill \\

  \item[\valobjtarget{}] \hfill \\


\end{description}
