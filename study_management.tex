\chapter{Study Management}

\section{Entities}

The entities in this module are:

\begin{description}

  \item[Study] \hfill \\ Represents a collection of participants and specimens
    collected for a particular research study. The study is part of an
    aggregate that define the types of specimens to be collected and how they
    are to be collected from participants.\par A study can be enabled or
    disabled. When disabled, changes to its configuration are possible but
    patients and specimens cannot be added to it. When enabled, no further
    configuration changes are allowed. However, once enabled participants and
    specimens can be added.\par  A study may have one or more comments assigned to
    it during it's lifetime.

  \item[CollectionEventType] \hfill \\ A uniquely named, within the \texttt{Study},
    classification of a visit by a \texttt{Patient}.

  \item[ProcessingType] \hfill \\ Describes a regularly performed procedure
    with a unique name (within its {@link Study}). There should be one or more
    associated {@link SpecimenProcesingLinkType}s that (1) further define legal
    procedures and (2) allow logging of procedures performed on different types
    of {@link Specimen}s.

  \item[] \hfill \\

  \item[] \hfill \\

  \item[] \hfill \\


\end{description}

\section{Value Objects}

The value objects in this module are:

\begin{description}

  \item[SpecimenGroupCollectionEventType] \hfill \\ Defines which types of
    specimens (i.e. which \texttt{SpecimenGroup}s) need to be collected as
    part of a \texttt{CollectionEventType}.

  \item[SpecimenLinkType] \hfill \\ Represents a regularly performed processing
    procedure involving two \texttt{Specimen}s: an input, which must be in a
    specific \texttt{SpecimenGroup}, and an output, which must be in a specific
    \texttt{SpecimenGroup}.

  \item[SpecimenGroup] \hfill \\ Ownership, summary, storage, and
    classification information that applies to an entire group or collection of
    \texttt{Specimen}s.

  \item[SpecimenType] \hfill \\ A standardised set of classifications that
    describe \emph{what} a \texttt{Specimen} is. Potential examples include:
    urine, whole blood, plasma, nail, protein, etc.

  \item[Preservation] \hfill \\ Describes how a \texttt{Specimen} should be
    preserved/stored by describing temperature requirements, as well as a
    preservation method (see \texttt{PreservationType}).

  \item[PreservationType] \hfill \\ A standardised set of methods for
    preserving and storing \texttt{Specimen}s.  Potential examples include:
    frozen specimen, RNA later, fresh specimen, slide, etc.

  \item[] \hfill \\

  \item[] \hfill \\

  \item[] \hfill \\

  \item[] \hfill \\


\end{description}
