\chapter{Study Management}

This section first describes the study aggregate, its composition, and then
defines the commands it handles.

The \entitytarget{Study} aggregate is used to configure different aspects for a
study. It is used to define the valid types of specimens that can be collected
from participants, when they are to be collected, how the collected specimens
are processed, and allows for customization with use of annotation types. This
aggregate is made up of the entities and value objects shown in the figure
below.

\begin{figure}[h]
  \includegraphics[trim={9mm 85mm 36mm 9mm}, clip,
    width=1\textwidth]{images/study-aggregate}
  \caption{Study aggregate}
  \label{fig:study-aggregate}
\end{figure}

\subsection*{Study}

A \entitylink{Study} represents a collection of participants and specimens
collected for a particular research study. Each study has a unique identifier,
\texttt{StudyId}, that is used to reference it. \texttt{name} is a short
descriptive name that is usually an acronym used for quick identification.
\texttt{description} give more details on the name and is usually the words
that make up the acronym.

A study can be enabled or disabled. When disabled, changes to its configuration
are possible but patients and specimens cannot be added. When enabled, no
further configuration changes are allowed, and participants and specimens can
be added.

As shown in the Figure \ref{fig:study-aggregate}, the study has collections of
other entities and value objects which are described below.

\subsection*{SpecimenGroup}

A \entitytarget{SpecimenGroup} is used to configure a specimen type used by the
study.  It records ownership, summary, storage, and classification information
that applies to an entire group or collection of \entitylink{Specimen}s. A
specimen group is defined either for specimens types collected from
participants, or for the types of specimens that are processed.

A specimen group has a name and a description: the name is a short identifying
name that is unique to the study, and the description can provide additional
details but is optional. The unit specifies how the specimen amount is measured
(e.g. volume, weight, length, etc.).

A study can have one or more specimen groups defined.  For specimen collection
to be allowed on a study, at least one specimen group must be defined.

The anatomical source, preservation medium and specimen type are defined using
other value objects discussed in Section \ref{sec:specimen-group}.

\subsection*{CollectionEventType}

A \entitytarget{CollectionEventType} defines a classification name, unique to
the \entitylink{Study}, for a visit by study participants. Each collection
event type is assigned one or more specimen groups to specify the specimen
types that are collected (see Section \ref{sec:collection-event-type}).

If the collection event of this type occurs more than once during the lifetime
of the study, then set \texttt{recurring} to \texttt{true}.

A study can have one or more collection event types defined. For specimen
collection to be allowed on a study, at least one collection event type must be
defined.

\subsection*{ProcessingType}

A \entitytarget{ProcessingType} describes a regularly performed specimen
processing procedure with a unique name (unique to the
\entitylink{Study}). There should be one or more associated
\entitylink{SpecimenLinkType}s that (1) further define legal procedures and (2)
allow recording of procedures performed on different types of
\entitylink{Specimen}s.

One or more processing types can be defined for a study. For specimen
processing to be allowed on a study, at least one processing type must be
defined.

\subsection*{StudyAnnotationType}

\entitytarget{StudyAnnotationType}s allows a study to collect custom named and
defined pieces of data on collection event types (Section
\ref{sec:collection-event-type}), processing events (Section
\ref{sec:processing-type}), and participants (Section
\ref{sec:participant-annotations}).

Annotations are optional and are not a requirement for specimen collection or
processing.

\subsection*{StudyComment}

A \entitytarget{StudyComment} contains a textual message and the user that
added the comment. The date and time the comment was made is recorded as meta
data. A study can have one or more comments.

\section{SpecimenGroup Details}
\label{sec:specimen-group}

The \entitylink{SpecimenGroup} entity is composed of the value objects shown
in Figure \ref{fig:specimen-group}.

\begin{figure}[h]
  \centering
  \includegraphics[trim={9mm 162mm 80mm 9mm}, clip,
    width=1\textwidth]{images/specimen-group}
  \caption{SpecimenGroup entity}
  \label{fig:specimen-group}
\end{figure}

\subsection*{AnatomicalSourceId}

An \valobjlink{AnatomicalSource} is a standardized set of regions from a
\entitylink{Participant} \emph{where} a \entitylink{Specimen} is collected
from. Potential examples include: colon, ear, leg, kidney,
etc. \valobjlink{AnatomicalSource} is a value object with a unique ID. They are
defined globally and new ones can be created at any time. They can be accessed
via look up service described in Section \ref{sec:lookup-service}.

A specimen group contains the ID of a single anatomical source.

\subsection*{PreservationId}

\valobjlink{Preservation} is a value object that describes how a
\entitylink{Specimen} should be preserved/stored by describing temperature
requirements ($^\circ$C), as well as a preservation method (see
\valobjlink{PreservationType}). \valobjlink{Preservation} is also a value
object with a unique ID. They are defined globally and new ones can be created
at any time. They can be accessed via look up service described in Section
\ref{sec:lookup-service}.

A specimen group contains the ID of a single preservation object.

\subsection*{SpecimenTypeId}

\valobjlink{SpecimenType} is standardized set of classifications that describe
\emph{what} a \entitylink{Specimen} is. Potential examples include: urine,
whole blood, plasma, nail, protein, etc. \valobjlink{SpecimenType} is also a
value object with a unique ID. They are defined globally and new ones can be
created at any time. They can be accessed via look up service described in
Section \ref{sec:lookup-service}.

A specimen group contains the ID of a single specimen type.

\section{CollectionEventType Details}
\label{sec:collection-event-type}

A collection even type can be configured to collect one or more specimens using
\entitylink{SpecimenGroupCollectionEventType}. It can also be configured to
record one or more annotation types using
\entitylink{CollectionEventTypeAnnotationType}. These associations are shown in
Figure \ref{fig:collection-event-type}.

\begin{figure}[H]
  \centering
  \includegraphics[trim={9mm 70mm 96mm 9mm}, clip,
    width=0.8\textwidth]{images/collection-event-type}
  \caption{Details for the CollectionEventType entity}
  \label{fig:collection-event-type}
\end{figure}

\subsection*{SpecimenGroupCollectionEventType}

\valobjtarget{SpecimenGroupCollectionEventType}s are used to define which types
of specimens (i.e. which \valobjlink{SpecimenGroup}s) need to be collected with
a type of collection event. A single specimen group can be used in multiple
collection event types.

The \texttt{count} specifies how many specimens are to be collected. The
\texttt{amount} is the amount of substance that is expected in each collected
specimen, or null if there is no default amount. The unit on the amount is
defined in the \entitylink{SpecimenGroup}.

\subsection*{CollectionEventAnnotationType}

Collection event annotations are defined using
\valobjtarget{CollectionEventAnnotationType}. One or more of these can be
defined for the study. A single collection event annotation type can be used in
multiple collection event types.

\section{ProcessingType Details}
\label{sec:processing-type}

A processing type can be configured to process one or more collected specimens
using \valobjlink{SpecimenLinkType}. Individual specimen link types within the
processing type can also be configured to record one or more annotation using
\valobjlink{SpecimenLinkTypeAnnotationType}. Figure \ref{fig:processing-type}
provides more details for the processing type entity.

\begin{figure}[H]
  \centering
  \includegraphics[trim={9mm 48mm 120mm 9mm}, clip,
    width=0.7\textwidth]{images/processing-type}
  \caption{Details for the ProcessingType entity.}
  \label{fig:processing-type}
\end{figure}

Processing of specimens is not allowed until at least one processing type is
defined for the study.

\subsection*{SpecimenLinkType}

 \valobjtarget{SpecimenLinkType}s are assigned to
a processing type, and used to represent a regularly performed processing
procedure involving two \entitylink{Specimen}s: an input, which must be in a
specific \valobjlink{SpecimenGroup}, and an output, which must be in a specific
\valobjlink{SpecimenGroup}.

The \texttt{expectedInputChange} is the expected amount (decimal value) to be
removed from each input. The \texttt{expectedOutputChange} (also a decimal
value) is the expected amount to be added to each output. If the expected input
and output change values are not required, they can be assigned a zero value.
The counts in \texttt{inputCount} and \texttt{outputCount} are the number of
expected and resulting specimens, respectively, when the processing is carried
out. A value of zero for output count implies that the count is the same as the
input count. The specimen container type that holds the input specimens is
given in \texttt{inputContainerType}. The specimen container type that the
output specimens are stored into is given in \texttt{outputContainerType}. If
specifying the container types is not required for one or both of these fields,
they can be assigned a \texttt{null} value.

To avoid redundancy, a specimen link type may exist only once for a given input
and output specimen group.

\subsection*{SpecimenLinkTypeAnnotationType}

A \valobjtarget{SpecimenLinkTypeAnnotationType} is used to tie a specimen
link annotation type to a specimen link type. When \texttt{required} is set to
\texttt{true} the annotation value is not allowed to be empty.

\section{Participant Annotations}
\label{sec:participant-annotations}

Figure \ref{fig:study-annotations} show the possible annotations types that can
be configured for a study. Annotations types must be defined before they can be
assigned to the respective entity.

\begin{figure}[H]
  \centering
  \includegraphics[trim={9mm 108mm 64mm 9mm}, clip,
    width=0.8\textwidth]{images/study-annotations}
  \caption{Study annotation entities}
  \label{fig:study-annotations}
\end{figure}

An annotation type has a short identifying name that is unique to the study. A
description can also be defined but is optional. The \texttt{valueType} can be
one of:

\begin{table}[!htbp]
\renewcommand{\arraystretch}{1.1}
\begin{tabularx}{\textwidth}{@{\hspace{6pt}} >{\ttfamily}l X }

  String & an alphanumeric string.\\
  Number & a number, either integer or decimal.\\
  Date & date string, usually of the form \emph{YYYY-MM-DD HH:MM}.\\
  Select & a value selected from a predefined list.\\

\end{tabularx}
\end{table}

When \texttt{valueType} is assigned to be of type \texttt{Select},
\texttt{maxValueCount} is the number of of items that can be selected from the
predefined list. If only one value is allowed, then \texttt{maxValueCount} has
a value of 1. If an unlimited number of values are allowed then,
\texttt{maxValueCount} has a value of 0.

\subsection*{CollectionEventTypeAnnotationType}

A \valobjtarget{CollectionEventTypeAnnotationType} is used to to configure an
annotation to be used by collection event types. When \texttt{required} is set
to true the annotation value is not allowed to be empty.

An example of an annotation on a collection even type can be consent given by
the participant at the collection visit.

\subsection*{SpecimenLinkAnnotationType}

A \valobjtarget{SpecimenLinkAnnotationType} is used to to configure an
annotation to be used with specimen link types. When \texttt{required} is set
to true the annotation value is not allowed to be empty.

An example of an annotation type on a specimen link can be the PMBC count on
the collected specimen.

\subsection*{ParticipantAnnotationType}

A \valobjtarget{ParticipantAnnotationType} is used to to configure a an
annotation for a participant. When \texttt{required} is set to true the
annotation value is not allowed to be empty.

An example of an annotation on a participant can be their gender.

\section {Study Aggregate Commands}

The commands handled by the study aggregate are listed below.

\subsection*{CreateStudyCommand}

Creates a new study. A study can be created at any time. If a study with the
given name already exists the command fails and throws a checked exception.

\begin{commandparmtable}

  name & string & A short descriptive name associated with the study. Usually
  an acronym.\\

  description & string & A more detailed name for the study. If an acronym is
  used for the study name, then the description should contain the words that
  make up the acronym.\\

\end{commandparmtable}

\subsection*{UpdateStudyCommand}

Update a study's name, description, or both. If a study with the
new name already exists the command fails and throws a checked exception.

\begin{commandparmtable}

  studyId & string & The study's unique identifier.\\

  name & string & The new name, or the original name if it's not being modified.\\

  description & string & The new description, or the original description if
  it's not being modified.\\

\end{commandparmtable}

\subsection*{EnableStudyCommand}

Enables a study. Once enabled the study is ready to collect and process
specimens from participants.

\begin{commandparmtable}

  studyId & string & The study's unique identifier.\\

\end{commandparmtable}

\subsection*{DisableStudyCommand}

Used to disable a study. A study may be disabled because no more specimen
collection and processing will be done on it, or because it requires
configuration changes. Once disabled specimen collection and processing cannot
be performed on this study.

\begin{commandparmtable}

  studyId & string & The study's unique identifier.\\

\end{commandparmtable}

\subsection*{CreateSpecimenGroupCommand}

Create a specimen group on a study. If the study is not disabled the command
fails and throws a checked exception. If a specimen group already exists with
the given name the command fails and throws a checked exception.

\begin{commandparmtable}

  studyId & string & The study's unique identifier.\\

  name & string & A short descriptive name.\\

  description & string & Provides more details for the specimen group. Can be left empty.\\

  anatomicalSourceId & string & The ID corresponding to the anatomical source this group
  belongs to.\\

  preservationId & string & The ID corresponding to the preservation used by this
  specimen group.\\

  specimenTypeId & string & The ID corresponding to the specimen type.\\

\end{commandparmtable}

\subsection*{UpdateSpecimenGroupCommand}

Updates a specimen group on a study. If the study is not disabled the command
fails and throws a checked exception. If a specimen group already exists with
the new name the command fails and throws a checked exception.

\begin{commandparmtable}

  studyId & string & The study's unique identifier.\\

  name & string & The new name, or the original name if it's not being modified.\\

  description & string & The new description, or the original description if
  it's not being modified.\\

  anatomicalSourceId & string & The ID corresponding to the updated or original
  anatomical source.\\

  preservationId & string & The ID corresponding to the updated or original preservation.\\

  specimenTypeId & string & The ID corresponding to the updated or original specimen type.\\

\end{commandparmtable}

\subsection*{DeleteSpecimenGroupCommand}

Deletes a specimen group from the study. If the study is not disabled the
command fails and throws a checked exception.

\begin{commandparmtable}

  studyId & string & The study's unique identifier.\\

  specimenGroupId & string & The specimen group's unique identifier.\\

\end{commandparmtable}

\subsection*{CreateCollectionEventType}

Create a collection event type for a study. If the study is not disabled the
command fails and throws a checked exception. If a collection event with the
given name already exists the command fails and throws a checked exception.

\begin{commandparmtable}

  studyId & string & The study's unique identifier.\\

  name & string & A short descriptive name.\\

  description & string & Provides more details. Can be left empty.\\

  recurring & boolean & Set to \texttt{true} if this collection event type
  occurs more than once for the duration of the study.\\

\end{commandparmtable}

\subsection*{UpdateCollectionEventType}

Used to update one or more attributes of a collection event type. If a collection event with the
given name already exists the command fails and throws a checked exception.

\begin{commandparmtable}

  studyId & string & The study's unique identifier.\\

  collectionEventTypeId & string & The collection event type's unique identifier.\\

  name & string & The new name, or the original name if it's not being modified.\\

  description & string & The new description, or the original description if
  it's not being modified.\\

  recurring & boolean & The new value, or the original value if it's not being
  modified.\\

\end{commandparmtable}

\subsection*{DeleteCollectionEventType}

Used to delete a collection event type.

\begin{commandparmtable}

  studyId & string & The study's unique identifier.\\

  collectionEventTypeId & string & The collection event type's unique identifier.\\

\end{commandparmtable}

\subsection*{}

%

\begin{commandparmtable}

\end{commandparmtable}

\subsection*{}

%

\begin{commandparmtable}

\end{commandparmtable}

\subsection*{}

%

\begin{commandparmtable}

\end{commandparmtable}


\subsection*{}

%

\begin{commandparmtable}

\end{commandparmtable}

\subsection*{}

%

\begin{commandparmtable}

\end{commandparmtable}

\subsection*{}


\subsection*{}

%

\begin{commandparmtable}

\end{commandparmtable}

\subsection*{}

%

\begin{commandparmtable}

\end{commandparmtable}

\subsection*{}

%

\begin{commandparmtable}

\end{commandparmtable}
%

\begin{commandparmtable}

\end{commandparmtable}

\subsection*{}

%

\begin{commandparmtable}

\end{commandparmtable}

\subsection*{}

%

\begin{commandparmtable}

\end{commandparmtable}

\subsection*{}

%

\begin{commandparmtable}

\end{commandparmtable}
