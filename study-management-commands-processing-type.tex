\subsection{Processing Type Commands}

\subsection*{CreateProcessingType}
Creates a processing type for a study. If the study is not disabled the command
fails and throws a checked exception. The command fails with a checked
exception if name already exists.

\begin{commandparmtable}

  studyId & string & The study's unique identifier.\\

  name & string & A short descriptive name.\\

  description & string & Provides more details. Can be left empty.\\

  enabled & boolean & Set to \compfont{true} when this processing type is still
  in use.\\

\end{commandparmtable}

\subsection*{UpdateProcessingType}
Updates one or more attributes of a processing type. If a processing with the
given name already exists the command fails and throws a checked exception.

\begin{commandparmtable}

  studyId & string & The study's unique identifier.\\

  processingTypeId & string & The processing type's unique identifier.\\

  name & string & The new name, or the original name if it's not being modified.\\

  description & string & The new description, or the original description if
  it's not being modified.\\

  enabled & boolean & The new value, or the original value if it's not being
  modified.\\

\end{commandparmtable}

\subsection*{DeleteProcessingType}
Deletes a processing type.

\begin{commandparmtable}

  studyId & string & The study's unique identifier.\\

  processingTypeId & string & The processing type's unique identifier.\\

\end{commandparmtable}

\subsection*{EnableProcessingType}
Used to enable a processing type that was previously disabled.

\begin{commandparmtable}

  studyId & string & The study's unique identifier.\\

  processingTypeId & string & The processing type's unique identifier.\\

\end{commandparmtable}

\subsection*{DisableProcessingType}
Used to disable a processing type that was previously enabled.

\begin{commandparmtable}

  studyId & string & The study's unique identifier.\\

  processingTypeId & string & The processing type's unique identifier.\\

\end{commandparmtable}

\subsection*{AddSpecimenLinkType}

\begin{commandparmtable}

  studyId & string & The study's unique identifier.\\

  processingTypeId & string & The processing type's unique identifier.\\

  expectedInputChange & decimal & Amount to be removed from input. Use a value
  of zero if amount change tracking is not required.\\

  expectedOutputChange & decimal & Amount to be added to each output. Use a value
  of zero if amount change tracking is not required.\\

  inputCount & integer & The number of expected specimens.\\

  outputCount & integer & The number of resulting specimens after processing. A
  value of zero implies that the output count is the same as the input count.\\

  inputContainerType & \entitylink{SpecimenContainer} & The specimen container type
  that holds the input specimens.\\

  outputContainerType & \entitylink{SpecimenContainer} & The specimen container type
  that holds the output specimens.\\

\end{commandparmtable}

\subsection*{UpdateSpecimenLinkType}

\begin{commandparmtable}

  studyId & string & The study's unique identifier.\\

  processingTypeId & string & The processing type's unique identifier.\\

  specimenLinkTypeId & string & The specimen link type's unique identifier.\\

  expectedInputChange & decimal & The new or original input change amount.\\

  expectedOutputChange & decimal & The new or original output change amount.\\

  inputCount & integer & The new or original number expected specimens.\\

  outputCount & integer & The new or original number of resulting specimens.\\

  inputContainerType & \entitylink{SpecimenContainer} & The new or original
  input specimen container type.\\

  outputContainerType & \entitylink{SpecimenContainer} & The new or original
  output specimen container type.\\

\end{commandparmtable}

\subsection*{RemoveSpecimenLinkType}

\begin{commandparmtable}

  studyId & string & The study's unique identifier.\\

  processingTypeId & string & The processing type's unique identifier.\\

  specimenLinkTypeId & string & The specimen link type's unique identifier.\\

\end{commandparmtable}

\subsection*{AddInputSpecimenGroupToSpecimenLinkType}

\begin{commandparmtable}

  studyId & string & The study's unique identifier.\\

  processingTypeId & string & The processing type's unique identifier.\\

  specimenLinkTypeId & string & The specimen link type's unique identifier.\\

  specimenGroupId & string & The specimen group's unique identifier.\\

\end{commandparmtable}

\subsection*{RemoveInputSpecimenGroupFromSpecimenLinkType}

\begin{commandparmtable}

  studyId & string & The study's unique identifier.\\

  processingTypeId & string & The processing type's unique identifier.\\

  specimenLinkTypeId & string & The specimen link type's unique identifier.\\

  specimenGroupId & string & The specimen group's unique identifier.\\

\end{commandparmtable}

\subsection*{AddOutputSpecimenGroupToSpecimenLinkType}

\begin{commandparmtable}

  studyId & string & The study's unique identifier.\\

  processingTypeId & string & The processing type's unique identifier.\\

  specimenLinkTypeId & string & The specimen link type's unique identifier.\\

  specimenGroupId & string & The specimen group's unique identifier.\\

\end{commandparmtable}

\subsection*{RemoveOutputSpecimenGroupFromSpecimenLinkType}

\begin{commandparmtable}

  studyId & string & The study's unique identifier.\\

  processingTypeId & string & The processing type's unique identifier.\\

  specimenLinkTypeId & string & The specimen link type's unique identifier.\\

  specimenGroupId & string & The specimen group's unique identifier.\\

\end{commandparmtable}

\subsection*{AddAnnotationToSpecimenLinkType}

\begin{commandparmtable}

  studyId & string & The study's unique identifier.\\

  processingTypeId & string & The processing type's unique identifier.\\

  specimenLinkTypeId & string & The specimen link type's unique identifier.\\

  specimenLinkAnnotationTypeId & string & The specimen link annotation type's
  unique identifier.\\

  required & boolean & Use \compfont{true} when this annotation type is required
  and cannot be left empty.\\

\end{commandparmtable}

\subsection*{RemoveAnnotationFromSpecimenLinkType}

\begin{commandparmtable}

  studyId & string & The study's unique identifier.\\

  processingTypeId & string & The processing type's unique identifier.\\

  specimenLinkTypeId & string & The specimen link type's unique identifier.\\

  specimenLinkAnnotationTypeId & string & The specimen link annotation type's
  unique identifier.\\

\end{commandparmtable}
% Local Variables:
% compile-command: "/usr/bin/rubber --pdf main"
% End:

