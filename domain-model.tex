\chapter{Domain Model Overview}

The Biobank application is composed of several modules. Each module is
introduced briefly here and more details are provided in the remaining
chapters.

\begin{description}

  \item[Study Management] \hfill \\ Specimens collected by the Biobank
    application belong to an individual studies. Before specimens can be
    recorded, the study must be defined, along with the specimen types being
    collected, how specimens are collected from the study's participants, and
    how they are processed. This modules manages these tasks.

  \item[Patient Collection] \hfill \\ This module is used to record specimens
    collected from participants belonging to a study. The specimens collected
    must be those configured for the study.

  \item[Center Configuration] \hfill \\ Centers participating in collecting,
    processing, or storing specimens and also centers that request processed
    specimens must be registered in the application. Centers have the
    flexibility to be configured in one or more of the allowed
    configurations. Eg. a center can be both a collecting site and a storage
    site.

  \item[Center Shipment] \hfill \\ Collecting centers ship specimens to a
    storage center for processing and storage. Storage centers ship processed
    specimens to request centers. This module is used to record shipments of
    specimens between centers.

  \item[Specimen Processing and Storage] \hfill \\ Processing centers process
    specimens collected from participants. Processing of specimens includes
    aliquoting, centrifuging, and combining of specimens with preservation
    fluids.  Once specimens have been processed, centers configured for storage
    can store them in their storage containers. This module is used to record
    the procesing of specimens and where specimens are stored.

  \item[Center Storage Configuration] \hfill \\ Storage centers have different
    types of storage containers. This module is used to define the layout of
    storage containers and also storage hierarchies.

  \item[Specimen Request] \hfill \\ After specimens have been processed and
    stored, request centers submit specimen requests to retrieve some or all of
    the specimens stored for a particular study. Specimen requests can be done
    for sets of participants or meet certain criteria. This module is used to
    aid in creating specimen requests and track shipping details.

\end{description}

The following chapters describe the entities, aggregates, command and events
used in the  modules listed above.
