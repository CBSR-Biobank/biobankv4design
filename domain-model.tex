\chapter{Domain Model Overview}

The Biobank application is composed of several modules. Each module is
introduced briefly here and more details are provided in the remaining
chapters.

\begin{description}

\item[Study Management] \hfill \\ Specimens collected by the Biobank
  application belong to individual \emph{study}. Before specimens can be
  recorded, the study must be defined, along with the specimen types being
  collected, how specimens are collected from the study's participants, and how
  they are processed. This modules manages these tasks.

\item[Specimen Collection] \hfill \\ This module is used to record
  specimens collected from participants belonging to a study. The specimens
  collected must be those configured for the study.

\item[Specimen Processing and Storage] \hfill \\ \emph{Processing centres}
  process specimens collected from participants. Processing of specimens
  includes aliquoting, centrifuging, and combining of specimens with
  preservation fluids.  Once specimens have been processed, \emph{storage
    centres} store them within storage containers. This module is used to
  record the processing of specimens and where specimens are stored.

\item[Centre Management] \hfill \\ Centres participating in collecting,
  processing, or storing specimens and also centres that request processed
  specimens must be registered in the application. Centres have the
  flexibility to be configured in one or more of the allowed
  configurations. E.g. a centre can be both a collecting site and a storage
  site.

\item[Centre Shipment] \hfill \\ A collecting centre ships specimens to a
  storage centre for processing and storage. A storage centre ships processed
  specimens to a request centre. This module is used to record shipments of
  specimens between centres.

\item[Centre Storage Management] \hfill \\ Storage centres have different
  types of storage containers. This module is used to define the layout of
  storage containers and also storage hierarchies.

\item[Specimen Request] \hfill \\ After specimens have been processed and
  stored, request centres submit orders for specimen to retrieve some or all of
  the specimens stored for a particular study. Specimen requests can be done
  for sets of participants or meet certain criteria. This module is used to aid
  in creating specimen requests and track shipping details.

\end{description}

The following chapters describe the entities, aggregates, command and events
used in the  modules listed above.

% Local Variables:
% compile-command: "/usr/bin/rubber --pdf main"
% End:

