\chapter{Specimen Collection}

The \entitytarget{Participant} aggregate records specimens collected
from study participants. Before specimen collection can be used, the
participant must be created. Participants cannot be added to a
\entitylink{Study} that is not enabled.

Figure \ref{fig:participant-aggregate} shows the composition of this aggregate.
A participant has a unique identifier that is used to identify the participant
in the system. This identifier is not the same as the
\valobjlink{ParticipantId} value object used by the domain model.

\begin{figure}[H]
  \centering
  \includegraphics[trim={10mm 67mm 98mm 18mm}, clip,
    width=0.75\textwidth]{images/participant-aggregate}
  \caption{Participant aggregate}
  \label{fig:participant-aggregate}
\end{figure}

\subsection*{CollectionEvent}
Represents a visit made by a \entitylink{Participant} during which
\entitylink{Specimen}s may have been collected and other information may have
been recorded. The \entitylink{CollectionEvent} entity is discussed in more
details below.

\subsection*{ParticipantAnnotation}
This entity holds the annotation value for the
\entitylink{ParticipantAnnotationType} defined in the study. See Chapter
\ref{chap:study-annotations} for how annotation values are recorded.

\subsection*{ParticipantComment}
A \entitytarget{ParticipantComment} contains a textual message and the user
that added the comment. The date and time the comment is made is recorded as
meta data. A participant can have one or more comments.

\section{CollectionEvent Details}
The \entitytarget{CollectionEvent} is used to record a visit by a participant
to a \entitylink{Centre} (e.g. a clinic). A collection event must have a
\entitylink{CollectionEventType} as defined by the study. See Section
\ref{sec:collection-event-type} for more details on the collection event type.

A collection event also has a \compfont{visitNumber} that is an integer and a
\compfont{timeDone} which is a time stamp for when the participant made the
visit to the centre. The format for the time stamp is: YYYY-MM-DD HH:MM.

\begin{figure}[H]
  \centering
  \includegraphics[trim={10mm 72mm 70mm 18mm}, clip,
    width=0.85\textwidth]{images/collection-event}
  \caption{CollectionEvent entity}
  \label{fig:collection-event}
\end{figure}

\subsection*{SpecimenCollectionEvent}
One or more specimens may have been collected from the participant. Each
specimen is recorded with the collection event. More information for the
\entitylink{Specimen} aggregate root is given in Chapter
\ref{chap:specimen-processing}.

\valobjtarget{SpecimenCollectionEvent} is used to associate a specimen to a
collection event. It is possible that a specimen has multiple collection events
for the case when specimens are transferred from one study to another. By
allowing multiple collection events, the information from the original
collection event is not lost for the specimen. It is not recommended that a
specimen have multiple collection events within the same study.

\subsection*{Specimen}
The \entitylink{Specimen} aggregate root is used to add specimens to the
system. It is discussed in more details in Section
\ref{sec:specimen-aggregate}.

\subsection*{CollectionEventAnnotation}
\entitytarget{CollectionEventAnnotation} holds the annotation value for the
\entitylink{CollectionEventAnnotationType} defined in the study. See
Chapter \ref{chap:study-annotations} for how annotation values are recorded.

\subsection*{CollectionEventComment}
A \entitytarget{CollectionEventComment} contains a textual message and the user
that added the comment. The date and time the comment is made is recorded as
meta data. A collection event can have one or more comments.

\section{Specimen Details}
\label{sec:specimen-aggregate}
\entitytarget{Specimen}s can be added to the system using the specimen
aggregate. A Specimen collected from a participant can be created with this
aggregate and then added to a collection event. Figure
\ref{fig:specimen-aggregate} shows that when a specimen is created it must be
assigned the corresponding \entitylink{SpecimenGroup} defined in the study.

\begin{figure}[H]
  \centering
  \includegraphics[trim={10mm 96mm 158mm 18mm}, clip,
    width=0.5\textwidth]{images/specimen-aggregate}
  \caption{Specimen Aggregate Root}
  \label{fig:specimen-aggregate}
\end{figure}

The \entitylink{Specimen} records the following: time the specimen was created;
the amount stored (the units are specified in \entitylink{SpecimenGroup}). The
location for the \entitylink{Centre} where the collection was done; the current
location for the specimen which may be different than the origin location; if
this specimen is stored then its location is recorded in the
\compfont{container} and \compfont{position} fields.

% Local Variables:
% compile-command: "/usr/bin/rubber --pdf main"
% End:

