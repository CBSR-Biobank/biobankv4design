\chapter{Specimen Collection}

The \entitytarget{Participant} aggregate records specimens collected
from study participants. Before specimen collection can be used, the
participant must be created. Participants cannot be added to a
\entitylink{Study} that is not enabled.

Figure \ref{fig:participant-aggregate} shows the composition of this aggregate.
A participant has a unique identifier that is used to identify the participant
in the system. This identifier is not the same as the
\valobjlink{ParticipantId} value object used by the domain model.


\begin{figure}[H]
  \centering
  \includegraphics[trim={10mm 75mm 102mm 10mm}, clip,
    width=0.75\textwidth]{images/participant-aggregate}
  \caption{Participant aggregate}
  \label{fig:participant-aggregate}
\end{figure}

\subsection*{CollectionEvent}
Represents a visit made by a \entitylink{Participant} during which
\entitylink{Specimen}s may have been collected and other information may have
been recorded.

\subsection*{ParticipantAnnotation}
This entity holds the annotation value for the
\entitylink{ParticipantAnnotationType} defined in the study. For a single
annotation, the \emph{Value Field} used (i.e. \compfont{stringValue},
\compfont{numberValue}, or \compfont{selectedValue}) depends on the annotation
type's \compfont{valueType} (see Section \ref{sec:study-annotations}). The
unused fields are assigned the \compfont{null} value.

\begin{table}[!htbp]
\renewcommand{\arraystretch}{1.1}
\begin{tabularx}{\textwidth}{l l}
  \sffamily{\textbf{ValueType}} & \sffamily{\textbf{Value field}}\\
  \hline
  String & \compfont{stringValue}\\
  Number & \compfont{numberValue}\\
  Date & \compfont{numberValue} and stored as the number of seconds\\
  Select & \compfont{selectedValue}\\

\end{tabularx}
\end{table}

\subsection*{ParticipantComment}
A \entitytarget{ParticipantComment} contains a textual message and the user
that added the comment. The date and time the comment was made is recorded as
meta data. A participant can have one or more comments.

\section{CollectionEvent Details}

The \entitytarget{CollectionEvent} is used to record a visit to a
\entitylink{Center} (e.g. a clinic) by one of the study's participants. A
collection event must have a type as defined for the study. See Section
\ref{sec:collection-event-type} for more details on the
\entitylink{CollectionEventType}.

A collection event also has a \compfont{visitNumber} that is an integer and a
\compfont{timeDone} which is a time stamp for when the participant made the
visit to the center. The format for the time stamp is: YYYY-MM-DD HH:MM.

\begin{figure}[H]
  \centering
  \includegraphics[trim={10mm 66mm 75mm 10mm}, clip,
    width=0.85\textwidth]{images/collection-event}
  \caption{CollectionEvent entity}
  \label{fig:collection-event}
\end{figure}

\subsection*{Specimen}
One or more specimens may have been collected from the participant. Each
specimen is recorded with the collection event. More information for this
entity is given in Section \ref{sec:specimen}.

\subsection*{CollectionEventAnnotation}
This entity holds the annotation value for the
\entitylink{CollectionEventTypeAnnotationType} defined in the study. This field
is similar to the \entitylink{ParticipantAnnotation}. See above for how it used.

\subsection*{CollectionEventComment}
A \entitytarget{CollectionEventComment} contains a textual message and the user
that added the comment. The date and time the comment was made is recorded as
meta data. A collection event can have one or more comments.


% Local Variables:
% compile-command: "/usr/bin/rubber --pdf main"
% End:

