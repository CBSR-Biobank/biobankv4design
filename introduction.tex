\chapter{Introduction}

Biobank version 4 is a rewrite of the Biobank application and meant to provide
the majority of its functionality through a web browser based interface. It
uses Domain Driven Design principles \cite{evans2004domain} and employs a CQRS
architecture (Command Query Responsibility Segregation)
\cite{vernon2013implementing}.

In addition, version 4 includes enhancements to the domain model which provides
better work flow and an improved user experience.

Flatbed scanning is supported by having a separate dedicated desktop client,
but this client's functionality is focused only on scanning and decoding tubes
etched with 2D barcodes.

\section{Dependencies}

This version of Biobank uses the following open source tools / software packages:

% table with no indentation
\begin{description}

  \item[Play Framework] \hfill \\ Play is an open source web application
    framework, written in Scala and Java, which follows the
    model–view–controller (MVC) architectural pattern. It aims to optimize
    developer productivity by using convention over configuration, hot code
    reloading and display of errors in the browser.

  \item[MongoDB ] \hfill \\ A document based NoSQL database and provides an
    event store implementation that stores event streams in a MongoDB database.

  \item[MySQL database ] \hfill \\ A SQL relational database management system
    used to store the query model.

  \item[Spring Framework ] \hfill \\ Provides a comprehensive programming and
    configuration model for modern Java-based enterprise applications.

  \item[Hibernate] \hfill \\ An object-relational mapping library that
    provides a framework to map a data model to a relational database.

  \item[Twitter Bootstrap ] \hfill \\ Bootstrap is a sleek, intuitive, and
    powerful front-end framework for faster and easier web user interface
    development \cite{bootstrap}

\end{description}

\section{Use of Domain Driven Design}

This document assumes the reader is familiar with Domain Driven Design. The
document uses it's terminology to describe the design of this application.

\section{CQRS Overview}

There are a number of reasons for choosing the CQRS architecture for
Biobank. Among these are:

\begin{itemize}
\item CQRS employs a message-based system that allows for decoupling and is more
  reliable against transient failures.
\item \emph{Commands} (that change data) are separated from \emph{queries}
  (that read data).
\item Enables domain driven design by capturing the intent in the form of
  commands.
\item \emph{Event Sourcing} allows persisting of application entities as a
  sequence of events that created them. Thus, captures all business changes in a
  loss less manner.
\item Events are used to populate views also known as the reporting side. Thus,
  allows for decoupling of domain logic from reporting.
\item Allows for definitions of new reports that provide new insight to the
  data that has already been captured.
\end{itemize}

\section{Using Scala versus Java}

The current version of Biobank, version 3, can be broken down into the
following layers:

\begin{enumerate}
\item User interface code on client side. This code uses the Eclipse RCP
  framework.
\item Client side code that communicates with the server. This code uses caCORE
  SDK as the underlying framework.
\item Server side code that communicates with the clients. This code also uses
  uses caCORE SDK as the underlying framework.
\item Server code that manages the domain model. This code uses Hibernate as
  the framework with a MySQL database.
\end{enumerate}

Another important item to note is that caCORE SDK only works with Java SDK
version 6 and older.

In Biobank version 4 the following changes are required:

\begin{itemize}
\item The user interface code must be rewritten to use a web based
  framework. Therefore, almost none of the version 3 code can be re-used for
  this layer.
\item For server and client communication, we no longer want to use caCORE SDK
  due to it being outdated, and would rather use more up to date
  technologies. By using more recent technologies, it will be easier to add
  more functionality (such as interworking with other applications). Also, we
  would like to incorporate Domain Driven Design principles. So, most of
  the code for layers 2 and 3 listed above must be refactored or completely
  rewritten.
\item In version 4, we would like to use CQRS(Command Query Responsibility
  Segregation) because it is a better DDD approach and to also meet regulatory
  requirements for audit logging.
\end{itemize}

At the time of writing this document, the work to update the domain model and
incorporate additional feature requirements has already been completed for
version 4. Due to the number of changes, it means that a small percentage of
the code for layer 4 above can be reused.

Given the requirements for version 4 a versy small amount of the current code
can be reused.  Due to this, re-writing the code in Scala is a sensible
approach and allows the use the latest available technologies in Web
development. Also, using Scala leads to, in most cases, less code. A module
that has 100 or more lines in Java can be rewritten in Scala with 20
lines. With less code to manage, the easier it is to maintain it. Scala has
gained much traction in web development in the last 2 years. Many big internet
sites such as Twitter, LinkedIn, etc., have already switched to Scala and use
the new tools provided that come with it.

% Local Variables:
% compile-command: "/usr/bin/rubber --pdf main"
% End:

