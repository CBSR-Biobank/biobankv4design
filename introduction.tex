\chapter{Introduction}

Biobank version 4 is a rewrite of the Biobank application, meant to provide the
majority of its functionality through a web browser based interface. It was
designed using Domain Driven Design principles \cite{evans2004domain} and
employs a CQRS architecture (Command Query Responsibility Segregation)
\cite{vernon2013implementing}.

In addition, version 4 includes enhancements to the domain model which provides
better workflow and an improved user experience.

Flatbed scanning is supported by having a separate dedicated client, but this
client's functionality is focused only on scanning and decoding tubes etched
with 2D barcodes.

\section{Dependencies}

This version of Biobank uses the following open source tools / software packages:

% table with no indentation
\begin{description}

  \item[Jetty Http Server] \hfill \\ The Embedded Jetty Server provides web
    browser with access to the application.

  \item[MongoDB ] \hfill \\ A document based NoSQL database and provides an
    event store implementation that stores event streams in a MongoDB database.

  \item[MySQL database ] \hfill \\ A SQL relational database management sytem
    used to store the query model.

  \item[Axon Framework ] \hfill \\ Provides the building blocks used in
    applying the CQRS architectural pattern.

  \item[Spring Framework ] \hfill \\ Provides a comprehensive programming and
    configuration model for modern Java-based enterprise applications.

  \item[Spring MVC ] \hfill \\ Model-view-controller framework for web
    applications.

  \item[Spring Security ] \hfill \\ A customizable authentication and
    access-control framework for securing web applications.

  \item[Hibernate ] \hfill \\ An object-relational mapping library that
    provides a framework to map a data model to a relational database.

  \item[Twitter Bootstrap ] \hfill \\ Bootstrap is a sleek, intuitive, and
    powerful front-end framework for faster and easier web user interface
    development \cite{bootstrap}

\end{description}

\section{Use of Domain Driven Design}

This document assumes the reader is familiar with Doman Driven Design. The
document uses many of it's terminology to describe the design of this
application.

\section{CQRS Overview}

There are a number of reasons for choosing the CQRS architecture for
Biobank. Amongst these are:

\begin{itemize}
\item Employs a message-based system that allows for decoupling and is more
  reliable against transient failures.
\item \textbf{Commands} (that change data) are separated from \textbf{queries}
  (that read data).
\item Enables domain driven design by capturing the intent in the form of
commands.
\item \textbf{Event Sourcing} allows persisting of application entities as a
  sequence of events that created them. Thus, captures all business changes in a
  lossless manner.
\item Events are used to populate views also known as the reporting side. Thus,
  allows for decoupling of domain logic from reporting.
\item Allows for definitions of new reports that provide new insight to the
  data that has already been captured.
\end{itemize}
