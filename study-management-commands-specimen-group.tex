\subsection{Specimen Group Commands}
\subsection*{CreateSpecimenGroupCommand}

Creates a specimen group on a study. If the study is not disabled the command
fails and throws a checked exception. If a specimen group already exists with
the given name the command fails and throws a checked exception.

\begin{commandparmtable}

  studyId & String & The study's unique identifier.\\

  name & String & A short descriptive name.\\

  description & String & Provides more details for the specimen group. Can be left empty.\\

  anatomicalSourceId & String & The ID corresponding to the anatomical source this group
  belongs to.\\

  preservationId & String & The ID corresponding to the preservation used by this
  specimen group.\\

  specimenTypeId & String & The ID corresponding to the specimen type.\\

\end{commandparmtable}

\subsection*{UpdateSpecimenGroupCommand}

Updates a specimen group on a study. If the study is not disabled the command
fails and throws a checked exception. If a specimen group already exists with
the new name the command fails and throws a checked exception.

\begin{commandparmtable}

  studyId & String & The study's unique identifier.\\

  name & String & The new name, or the original name if it's not being modified.\\

  description & String & The new description, or the original description if
  it's not being modified.\\

  anatomicalSourceId & String & The ID corresponding to the updated or original
  anatomical source.\\

  preservationId & String & The ID corresponding to the updated or original preservation.\\

  specimenTypeId & String & The ID corresponding to the updated or original specimen type.\\

\end{commandparmtable}

\subsection*{DeleteSpecimenGroupCommand}

Deletes a specimen group from the study. If the study is not disabled the
command fails and throws a checked exception.

\begin{commandparmtable}

  studyId & String & The study's unique identifier.\\

  specimenGroupId & String & The specimen group's unique identifier.\\

\end{commandparmtable}
% Local Variables:
% compile-command: "/usr/bin/rubber --pdf main"
% End:

