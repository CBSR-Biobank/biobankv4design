\section{Commands}

The commands for adding a collection event type - annotation type are shown
here. The commands to add annotations to \entitylink{SpecimenLink} and
\entitylink{Participants} as similar to these commands but are not listed here.

See the \entitytarget{AddAnnotationOption} command for adding annotation
options to \compfont{SELECT} annotation types.

\subsection*{CreateCollectionEventTypeAnnotationType}

\begin{commandparmtable}
  studyId & string & The study's unique identifier.\\

  name & string & A short descriptive name.\\

  description & string & Provides more details. Can be left empty.\\

  valueType & \valobjlink{AnnotationValueType} & The types of values
  (e.g. string, number, date, etc.) this type of annotation expects.\\

  maxValueCount & integer & If value type is \compfont{SELECT}, then this is the
  maximum number of selections that can be made. Use zero for unlimited.\\
\end{commandparmtable}

\subsection*{UpdateCollectionEventTypeAnnotationType}

\begin{commandparmtable}
  studyId & string & The study's unique identifier.\\

  collectionEventAnnotationTypeId & string & The annotation type's unique identifier.\\

  name & string & The updated or original name.\\

  description & string & The updated or original description.\\

  valueType & \valobjlink{AnnotationValueType} & The updated or original value type.\\

  maxValueCount & integer & The updated or original count.\\
\end{commandparmtable}

\subsection*{DeleteCollectionEventTypeAnnotationType}

\begin{commandparmtable}
  studyId & string & The study's unique identifier.\\

  collectionEventAnnotationTypeId & string & The annotation type's unique identifier.\\
\end{commandparmtable}

\subsection*{AddCollectionEventTypeAnnotationOptions}
The command fails if the annotation value type is not \compfont{SELECT}.

\begin{commandparmtable}
  studyId & string & The study's unique identifier.\\

  collectionEventAnnotationTypeId & string & The annotation type's unique
  identifier.\\

  options & Set[String] & The options to add for this annotation.\\
\end{commandparmtable}
% Local Variables:
% compile-command: "/usr/bin/rubber --pdf main"
% End:

